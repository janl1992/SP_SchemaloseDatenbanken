% Document
\chapter*{Beispielgliederung}
%    \section*{beispielgliederung}\markboth{beispielgliederung}{}
%    \addcontentsline{toc}{chapter}{beispielgliederung}
    \begin{enumerate}
        \item Graph-Datenbanken - Grundlegende technologische Aspekte
        \begin{enumerate}[label*=\arabic*.]
            \item Einführung
            \item Modell (Graph, Property Graphen, Hypergraphen)
        \end{enumerate}
        \item Graph-Datenbanken und -Frameworks - PostgresSQL
        \begin{enumerate}[label*=\arabic*.]
            \item Allgemein
            \item Architektur
            \item Datenmodell
            \item Indexe
            \item Anfragemethoden
            \item Konsistenz
        \end{enumerate}
        \item Graph-Datenbanken im praktischen Einsatz: OLTP
            \begin{enumerate}[label*=\arabic*.]
                \item Ausgewählte Use Cases
                \item Weitere Zugriffstechniken
                \item Vergleich mit relationalen Datenbanksystemen
                \item Beurteilung
            \end{enumerate}
        \item Graph-Datenbanken im praktischen Einsatz: OLAP
        \begin{enumerate}[label*=\arabic*.]
            \item Benchmark
            \item Datenbankzugriffe
            \item Zugriffsart Aggregation
            \item Zugriffsart Traversierung
            \item Interpretation der Ergebnisse
        \end{enumerate}
    \end{enumerate}

%\end{beispielgliederung}
\begin{abstract}
\section*{Exposé}\markboth{Exposé}{}
  \addcontentsline{toc}{chapter}{Exposé}
  

Überbegriff	Inhalt: Was soll im Exposé geleistet werden?
\begin{itemize}
\item Problemstellung	„Welches theoretische, praktische Problem ist Ausgangspunkt der Arbeit?“
\begin{itemize}
	\item Es soll untersucht werden wie sich Graphen in der relationalen Datenbank Postgresql abbilden lassen
	\item Ist es besser eine Relationale Datenbank als Graphdatenbank zu verwenden oder eine "richtige" Graphendatenbank zu verwenden.
	\item Welche Problemstellungen lassen sich durch die Modellierung von Daten in Form von Graphen lösen
	\item Optimierungemöglichkeiten? 
\end{itemize}
\item Forschungsstand	Eigene Vorkenntnisse im Exposé darlegen: Wie ist der aktuelle Forschungsstand zum Thema der Arbeit?
\begin{itemize}
	\item ... (Kein Dunst) Ist das für uns überhaupt relevant?
\end{itemize}
\item Wissenslücke/Erkenntnisinteresse	Warum will man sich mit diesem Problem beschäftigen? Welche Wissenslücke kann man dadurch schließen?
 \begin{itemize}
 	\item Praktische Anwendungsgebiete wie Empfehlungsengines o.Ä.
 	\item Vorteile von Relationalen Datenbanken
 	\item Auswirkungen auf bestehende Anwendungen
 \end{itemize}
\item Fragestellung	Welche Frage (Forschungsfrage) soll in der Arbeit beantwortet werden?
\begin{itemize}
	\item ... (Kein Dunst) Haben wir überhaupt eine Forschungsfrage?
\end{itemize}
\item Ziel/Hypothese	Im Exposé darlegen, welches Ziel erreicht werden soll: Was soll bewiesen oder widerlegt werden? Kurz: „Was will ich wissen? (Fragestellung), wozu will ich das wissen? (Ziel)“
\begin{itemize}
	\item Wie performant lässt sich die Traversierung mit hilfe von Stored Procedures abbilden
	\item Welche Besonderheiten bietet Postgresql. 
\end{itemize}
\item Theoriebezug	Welche Theorien sollen als Basis für die Bearbeitung der Fragestellung dienen?
\begin{itemize}
	\item Graphentheorie
	\item Relationale Algebra
\end{itemize}
\end{itemize}
 
\end{abstract}

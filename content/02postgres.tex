\chapter{Graph-Datenbanken und -Frameworks - Ausgewählte Systeme }
\section{PostgreSQL}
PostgreSQL ist eine Objektrelationale Datenbank. Weiterentwicklungen wird von der PostgreSQL Global Development Group durchgeführt. PostgreSQL steht unter der PostgreSQL-Lizenz, welche sehr stark der GNU-Lizenz ähnelt. Auf Basis von PostgreSQL gibt es mehrere, zum Teil auch kommerzielle, Forks wie zum Beispiel Amazon Redshift oder EnterpriseDB.
\section{Allgemein}
\newpage
\section{Architektur}
Eine einzelne PostgreSQL-Instanz mit einer oder mehreren Datenbanken wird auch Cluster genannt. Ein PostgreSQL-Cluster ist ein eigener Prozess mit einem eigenen Datenverzeichnis eigenen Konfigurationsdatei sowie einem eigenen Transaktionslog.
Die Architektur von PostgresSQL ist in folgendem Bild gegeben:
\begin{center}
    \includegraphics[width = \linewidth]{./images/PostgresSQLArchitektur.jpg}
\end{center}
Die Hauptaufgabe des Shared Buffer ist es, Input Output Operationen (I/O Operationen) auf das Datafile zu minimieren und möglichst viele Operationen im Speicher durchzuführen. Die Motivation möglichst viele Operation im Speicher durchzuführen
besteht darin, dass Operationen im Speicher schneller ausgeführt werden. Werden Operationen im Speicher ausgeführt, so werden I/O Operationen auf das Datafile reduziert. Der Writer Prozess ist für die Synchronisation
des Zustands des Shared Buffers und der Tablespaces verwantwortlich. Dieser Prozess schreibt Datenblöcke aus dem Shared Buffer auf die Tablespaces innerhalb des Datafile. Der Writer Checkpoint ist dafür verantwortlich,
dass alle geänderten Datenblöcke innerhalb des Shared Buffer in das Datafile geschrieben werden. \footnote{Vgl. \cite[Seite 26]{froehlich01}} \\
Eine PostgreSQL-Instanz wird als Server Prozess mit eigenem Datenverzeichnis und einer eigenen Konfigurationsdatei sowie einem eigenen Transaktionslog.
\section{Datenmodell}
. Daten werden in Form von Tabellen abgelegt. Im Filesystem legt PostgreSQL Dateien im sogenannten \$PGDATA-Verzeichnis ab, dieses wird beim Start von PostgreSQL festgelegt. Alle Date werden im Verzeichnis Base unterhalb von \$PGDATA abgelegt. Einzelne Tabellen oder Indices können in Tablespaces ausgelagert werden. Für einen Tablespace wird ein neuer Unterordner im \$PGDATA-Verzeichnis angelegt. Es ist auch möglich einzelne Spalten einer Tabelle in einen anderen Tablespace zu verschieben.
\section{Indexe}
\section{Anfragemethoden}
\section{Konsistenz}
PostgreSQL schreibt einen Transaktionslog (Write-Ahead Log, WAL).
Dieser wird als WAL-Buffer im Arbeitsspeicher und als WAL-Dateien auf der Festplatte geführt.
Bei jedem Commit einer Transaktion wird zunächst das WAL aktualisiert bevor die Bestätigung an den Client gesendet wird.
Der walwriter-Prozess schreibt periodisch die WAL-Buffer auf die Festplatte.

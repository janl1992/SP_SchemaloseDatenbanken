\chapter{Graph-Datenbanken und -Frameworks - Ausgewählte Systeme }
\section{PostgreSQL}
PostgreSQL ist eine Objektrelationale Datenbank. Weiterentwicklungen wird von der PostgreSQL Global Development Group durchgeführt. PostgreSQL steht unter der PostgreSQL-Lizenz, welche sehr stark der GNU-Lizenz ähnelt. Auf Basis von PostgreSQL gibt es mehrere, zum Teil auch kommerzielle, Forks wie zum Beispiel Amazon Redshift oder EnterpriseDB.
\section{Allgemein}
\section{Architektur}
Eine einzelne PostgreSQL-Instanz mit einer oder mehreren Datenbanken wird auch Cluster genannt. Ein PostgreSQL-Cluster ist ein eigener Prozess mit einem eigenen Datenverzeichnis eigenen Konfigurationsdatei sowie einem eigenen Transaktionslog.
\section{Datenmodell}
. Daten werden in Form von Tabellen abgelegt. Im Filesystem legt PostgreSQL Dateien im sogenannten \$PGDATA-Verzeichnis ab, dieses wird beim Start von PostgreSQL festgelegt. Alle Date werden im Verzeichnis Base unterhalb von \$PGDATA abgelegt. Einzelne Tabellen oder Indices können in Tablespaces ausgelagert werden. Für einen Tablespace wird ein neuer Unterordner im \$PGDATA-Verzeichnis angelegt. Es ist auch möglich einzelne Spalten einer Tabelle in einen anderen Tablespace zu verschieben.
\section{Indexe}
\section{Anfragemethoden}
\section{Konsistenz}
PostgreSQL schreibt einen Transaktionslog (Write-Ahead Log, WAL).
Dieser wird als WAL-Buffer im Arbeitsspeicher und als WAL-Dateien auf der Festplatte geführt.
Bei jedem Commit einer Transaktion wird zunächst das WAL aktualisiert bevor die Bestätigung an den Client gesendet wird.
Der walwriter-Prozess schreibt periodisch die WAL-Buffer auf die Festplatte.

\chapter{Graph-Datenbanken und -Frameworks - Ausgewählte Systeme }
\section{PostgreSQL}
\subsection{Allgemein}
    \begin{itemize}
        \item Kategorie / Modell
            \subitem PostgreSQL ist ein Relationales Datenbank System \cite{postgresqldoc}
        \item Version
            \subitem Aktuelle Version: 11
        \item Historie
            \subitem PostgreSQL ist aus dem POSTGRES projekt der University of California at Berkeley entstanden, welches unter der Leitung von  Professor Michael Stonebraker im Jahre 1986 began.
            SQL Interpreter seit 1994, das System hieß zu diesem Zeitpunkt Postgres95. 1996 wurde es in PostgrSQL umbenannt.\cite{postgresqldoc}
        \item Hersteller
        \item Lizenz
            \subitem Open-Source
    \end{itemize}
\subsection{Architektur}
    \begin{itemize}
        \item Programmiersprache des Systems
        \item Systemkomponenten, Systemarchitektur
        \subitem client / server Architektur \cite{postgresqldoc}
        \item Betriebsart
            \subitem Cluster - Menge an Datenbanken, die von PostSQL-Server verwaltet werden \cite{froehlich01}
        \item Protokoll der Schnittstelle
        \subitem TCP/IP
        \item API
    \end{itemize}

\subsection{Datenmodell}
\subsection{Indexe}
\subsection{Anfragemethoden}
\subsection{Konsistenz}
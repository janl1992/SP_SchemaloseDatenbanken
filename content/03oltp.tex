\chapter{Graph-Datenbanken im praktischen Einsatz: OLTP}
\section{PostgresSQL: OLTP}
\subsection{Ausgewählte Use Cases}
\subsection{CSV-Import}
Beim Import von (CSV)-Dateien wird zwischen import vom Clientsystem und  Import vom Serversystem unterschieden. 
Für den Import vom Client wird das psql-Statement \textbackslash copy verwendet. \textbackslash copy liest Informationen aus einer Datei die vom psql-Client aus erreichbar sein muss. \cite{postgres2018}

\begin{lstlisting}[language=SQL,caption=CSV Input,frame=single]
\copy Beitraege 
FROM './data/Beitraege.csv' DELIMITER ',' CSV HEADER;
\end{lstlisting}


\subsection{Beurteilung}
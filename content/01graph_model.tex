\chapter{Graph-Datenbanken - Grundlegende technologische Aspekte}
\section{Einführung}
\section{Modell}
\subsection{Graph}
Ein Graph G besteht aus einer nichtleeren Menge an Knoten V und Kanten E (G = (V,E)).
Die Kanten stellen die Beziehung zwischen den einzelnen Knoten her.
In einem klassischen Graphen kann eine Kante immer nur jeweils zwei Knoten miteinander verbinden.
Graphen können gerichtet oder ungerichtet sein. Gerichtete Graphen zeichnen sich dardurch aus, dass die Kanten eine zugewiesen Richtung besitzen.
Um die Beziehung zwischen zwei Knoten genauer zu definieren lassen sich die Kanten gewichten.
In diesem Fall werden den Kannten in der Regel nummerische Werte zugeordnet und man bezeichnet diese Graphen als Gewichtete Graphen.
Werden Kanten und Knoten eines Graphs vertauscht entsteht der Kantengraph bzw. Line-Graph des jeweiligen Graphen L(G).
Zwei Graphen können isomorph sein.
\\Verschiedene Graphen:
\begin{itemize}
	\item Planare Graphen
	\subitem lassen sich in der Ebene ohne Überschneidung der Kanten zeichnen\cite{Theobald2016}
	\item Reguläre Graphen
	\subitem alle Knoten haben den selben Knotengrad
	\item Partite Graphen
	\subitem die Knoten können in verschiedene Partitionen unterteilt werden
	\item Multimodale Graphen
\end{itemize}

%\\Ein Graph ist mathematisch folgendermaßen definiert: \\
%\begin{definition}
%	Ein $\text{Graph(graph)}G=(V,E,\gamma)$ ist ein Tripel bestehend aus:
%	\begin{itemize}
%		\item $V$, einer nicht leeren Menge von Knoten(vertices)
%		\item $E$, einer Menge von Kanten (edges) und
%		\item $\gamma$ , einer Inzidenzabbildung (incidence relation), mit\\
%		$\gamma : E \longrightarrow \{X | X \subseteq V, 1 \leq |X| \leq 2\}$
%	\end{itemize}
%	Zwei Knoten $a,b \in V$ heißen adjazent(adjacent) genau dann wenn
%	$\exists e \in E: \gamma(e)=\{a,b\}$. \\
%	Ein Knoten $a \in V$ und eine Kante $e \in E$ heißen inzident (incident)
%	genau dann wenn $a \in \gamma(e)$. \footnote{Vgl. \cite[Seite 21]{pbeck01}}
%\end{definition}
\subsection{Property Graphen}
Property Graphen zeichnen sich durch ihre den Kanten und Knoten zugewiesenen Eigenschaften auf.
\subsection{Hypegraphen}
Hypergraphen haben die Eigenschaft, dass Kanten im Gegensatz zu klassischen Graphen mehr als zwei Knoten miteinander verbinden können.
\\Mathematisch ist ein Hypergraph ist folgendermaßen definiert:
\begin{definition}
	Let $X=\{x_{1}, x_{2},...,x_{n}\}$ be a finite set,
	and let $E=\{e_{1},e_{2},...,e_{m}\}$ be a family of subsets of $X$ such that
	\[e_{i} \neq \varnothing (i=1,2,...,m) \\
	\cup_{i=1}^{m}e_{i}=X.
	\]
	The pair $H=(X,E)$ is called a hypergraph with vertex set $X$
	and hyperedge set $E$. The elements $x_{1}, x_{2},...,x_{n}$ of $X$ are vertices
	of hypergraph $H$, and the sets $e_{1}, e_{2},...,e_{m}$ are hyperedges of hypergraph $H$.\footnote{Vgl. \cite[Seite 2]{zhang2018hypergraph}}
\end{definition}

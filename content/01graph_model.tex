\chapter{Graph-Datenbanken - Grundlegende technologische Aspekte}
%\section{Einführung}
\section{Modell}
\subsection{Graph}
%Ein Graph G besteht aus einer nichtleeren Menge an Knoten V und Kanten E.
Ein Graph ist mathematisch folgendermaßen definiert:
\begin{definition}
	Ein $\text{Graph}G=(V,E,\gamma)$ ist ein Tripel bestehend aus:
	\begin{itemize}
		\item $V$, einer nicht leeren Menge von Knoten(vertices)
		\item $E$, einer Menge von Kanten (edges) und
		\item $\gamma$ , einer Inzidenzabbildung (incidence relation), mit\\
		$\gamma : E \longrightarrow \{X | X \subseteq V, 1 \leq |X| \leq 2\}$
	\end{itemize}
	Zwei Knoten $a,b \in V$ heißen adjazent(adjacent) genau dann wenn
	$\exists e \in E: \gamma(e)=\{a,b\}$. \\
	Ein Knoten $a \in V$ und eine Kante $e \in E$ heißen inzident (incident)
	genau dann wenn $a \in \gamma(e)$. \footnote{\cite[Seite 21]{pbeck01}}
\end{definition}
Die Kanten stellen die Beziehung zwischen den einzelnen Knoten her.
In einem klassischen Graphen kann eine Kante immer nur jeweils zwei Knoten miteinander verbinden.
Graphen können gerichtet oder ungerichtet sein. Gerichtete Graphen zeichnen sich dardurch aus, dass die Kanten eine zugewiesen Richtung besitzen.
Um die Beziehung zwischen zwei Knoten genauer zu definieren lassen sich die Kanten gewichten.
In diesem Fall werden den Kannten in der Regel nummerische Werte zugeordnet und man bezeichnet diese Graphen als Gewichtete Graphen.
Werden Kanten und Knoten eines Graphs vertauscht entsteht der Kantengraph bzw. Line-Graph des jeweiligen Graphen L(G).
Zwei Graphen können isomorph sein.

\subsection{Reguläre Graphen}
%@ToDo Bilder einfügen
Bei Reguläre Graphen haben alle Knoten den selben Knotengrad.
\subsection{Planare Graphen}
%@ToDo Bilder einfügen
Planare Graphen lassen sich in der Ebene ohne Überschneidung der Kanten zeichnen\cite{Theobald2016}
\subsection{Property Graphen}
%@ToDo Bilder einfügen
Property Graphen zeichnen sich durch ihre den Kanten und Knoten zugewiesenen Eigenschaften auf.
\subsection{k-Partite Graphen}
%@ToDo Bilder einfügen
Die Knoten können in k Partitionen unterteilt werden.
\subsection{Hypegraphen}
Hypergraphen haben die Eigenschaft, dass Kanten im Gegensatz zu klassischen Graphen mehr als zwei Knoten miteinander verbinden können.
\\Mathematisch ist ein Hypergraph ist folgendermaßen definiert:
\begin{definition}
	Let $X=\{v_{1}, v_{2},...,v_{n}\}$ be a finite set,
	and let $E=\{e_{1},e_{2},...,e_{m}\}$ be a family of subsets of $X$ such that
	\[e_{i} \neq \varnothing (i=1,2,...,m) \\
	\cup_{i=1}^{m}e_{i}=X.
	\]
	The pair $H=(X,E)$ is called a hypergraph with vertex set $X$
	and hyperedge set $E$. The elements $v_{1}, v_{2},...,v_{n}$ of $X$ are vertices
	of hypergraph $H$, and the sets $e_{1}, e_{2},...,e_{m}$ are hyperedges of hypergraph $H$.\footnote{Vgl. \cite[Seite 2]{zhang2018hypergraph}}
\end{definition}
Das folgende Bild zeigt einen Hypergraphen:
\begin{center}
	\includegraphics[scale = 0.5]{./images/Hypergraph.jpg}
\end{center}
Im Vergleich zum normale Graphen können die Kanten eines Hypergraphen eine beliebige Kardinalität haben (siehe Definition Definition Graph Kapitel 1.2.1) . Beim normalen Graphen können die Kanten nur die Kardinalität $1 \leq |X| \leq 2$
haben. Die Hyperedges in einem Hypergraphen sind somit eine beliebige Menge von Knoten. In einem normale Graphen sind die Kanten, eine in einem Intervall festgelegte Menge von Knoten:
    \[X = \{v_{1}, v_{2}, v_{3}, v_{4}, v_{5}, v_{6}, v_{7}\} \text{ Knoten}\]
    \[E=\{e_{1}, e_{2}, e_{3}, e_{4}\} \text{ Kanten}\]
    \[E=\{e_{1}, e_{2}, e_{3}, e_{4}\} = \{\{v_{1}, v_{2}, v_{3}\}, \{v_{2}, v_{3}\}, \{v_{3}, v_{5}, v_{6}\}, \{v_{4}\}\} \]
\subsection{Multimodale Graphen}
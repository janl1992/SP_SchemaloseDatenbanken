\chapter{Graph-Datenbanken - Grundlegende technologische Aspekte}
\section{Einführung}
\section{Modell}
\subsection{Graph}
Ein Graph ist mathematisch folgendermaßen definiert: \\
\begin{definition}
	Ein $\text{Graph(graph)}G=(V,E,\gamma)$ ist ein Tripel bestehend aus:
	\begin{itemize}
		\item $V$, einer nicht leeren Menge von Knoten(vertices)
		\item $E$, einer Menge von Kanten (edges) und
		\item $\gamma$ , einer Inzidenzabbildung (incidence relation), mit\\
		$\gamma : E \longrightarrow \{X | X \subseteq V, 1 \leq |X| \leq 2\}$
	\end{itemize}
	Zwei Knoten $a,b \in V$ heißen adjazent(adjacent) genau dann wenn
	$\exists e \in E: \gamma(e)=\{a,b\}$. \\
	Ein Knoten $a \in V$ und eine Kante $e \in E$ heißen inzident (incident)
	genau dann wenn $a \in \gamma(e)$. \footnote{Vgl. \cite[Seite 21]{pbeck01}}
\end{definition}
\subsection{Property Graphen}
Property Graphen zeichnen sich durch ihre den Kanten und Knoten zugewiesenen Eigenschaften auf.
\subsection{Hypegraphen}
Ein Hypergraph ist folgendermaßen definiert:
\begin{definition}
	Let $X=\{x_{1}, x_{2},...,x_{n}\}$ be a finite set,
	and let $E=\{e_{1},e_{2},...,e_{m}\}$ be a family of subsets of $X$ such that
	\[e_{i} \neq \varnothing (i=1,2,...,m) \\
	\cup_{i=1}^{m}e_{i}=X.
	\]
	The pair $H=(X,E)$ is called a hypergraph with vertex set $X$
	and hyperedge set $E$. The elements $x_{1}, x_{2},...,x_{n}$ of $X$ are vertices
	of hypergraph $H$, and the sets $e_{1}, e_{2},...,e_{m}$ are hyperedges of hypergraph $H$.\footnote{Vgl. \cite[Seite 2]{zhang2018hypergraph}}
\end{definition}

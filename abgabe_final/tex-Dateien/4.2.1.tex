\section{PostgreSQL}
Im folgenden Abschnitt werden die in Kapitel 3 vorgestellten SQLs auf ihre Leistungsfähigkeit untersucht. Für die Messung wurden vier verschiedene Stored Procedures angelegt:
\begin{itemize}
	\item innerJoinGenerator
	\item recursivesearch
	\item selectCascadingGenerator
	\item selectUnionGenerator
\end{itemize}
Der Aufruf der Statements funktioniert gleich, es werden die Rekursionstiefe, der Startknoten und die Tabelle, auf welcher die Stored Procedure ausgeführt wird, übergeben.

Die Funktionen innerJoinGenerator, selectCascadingGenerator und selectWithUnionSourceCodeGenerator generieren die entsprechenden Statements und führen diese aus. Die Funktion innerJoinGenerator erzeugt ein Select Statement in der die Abfrage, wie in Abschnitt \ref{2.postgresInnerJoin.subsection} beschrieben, erzeugt und ausgeführt wird. Entsprechend erzeugt selectCascadingGenerator eine verschachtelte Select-Abfrage. Die Funktion ist in Listing \ref{2.StandardSQLGenerisch.listing} abgebildet. Die Funktion selectUnionGenerator erzeugt eine Abfrage swie in Abschnitt \ref{2.postgresStandardSQL.subsection} beschrieben. Bei der Funktion recursivesearch handelt es sich um die Funktion wie sie in Abschnitt \ref{2.postgresRecursiveFunction.subsection} beschrieben.

\subsection{Graphtraversierung mit Hilfe von selbstgeschriebenen Stored Procedure}
\label{2.postgresRecursiveFunction.subsection}
Der Graph wird durch eine selbst erstellte Stored Procedure, die sich selber bis zu einer mitgegebenen Rekursionstiefe wieder aufruft, traversiert (siehe \ref{2.recursiveFunction.listing}).
Die \textcolor{red}{Abbruchbedingung} wird dem Stored Procedure in Form einer Rekursionsstiefe mitgegeben.
\begin{lstlisting}[language=SQL,caption = Abbruchbedingung recursiveSearch,frame=single, label={2.AbbruchbedingungRecursiveSearch.listing} ]
    CREATE OR REPLACE FUNCTION recursivesearch(tInput integer[], <@\textcolor{red}{iRecursionDepth integer} @>, sTable text) RETURNS SETOF integer AS
\end{lstlisting}
In jeder Rekursionsstufe erstellt das Skript zwei temporäre Tabelle.
Eine \textcolor{blue}{temporäre Tabelle} wird auf Basis eines \textcolor{red}{Eingabeparameters} (Datenstruktur Array) mit Hilfe des \textcolor{orange}{unnest Operators} erstellt.
\begin{lstlisting}[language=SQL,caption = Signatur recursiveSearch,frame=single, label={2.AbbruchbedingungRecursiveSearch.listing} ]
    <@\textcolor{blue}{CREATE TEMPORARY TABLE} @> intermDst AS SELECT * FROM <@\textcolor{orange} {unnest} @> <@\textcolor{red}{(tInput)}@>;
\end{lstlisting}
Diese temporäre Tabelle besitzt nur eine Spalte.
Diese Tabelle stellt die Spalte $src$ der aktuellen Rekursionsstufe dar.
Sie wird im \textcolor{red}{$IN$} Operator der $WHERE$ Klausel verwendet, um die \textcolor{blue}{zweite temporäre Tabelle} zu erstellen.
\begin{lstlisting}[language=SQL,caption = Erstellen 2. temporäre Tabelle,frame=single, label={2.AbbruchbedingungRecursiveSearch.listing} ]
    <@\textcolor{blue}{EXECUTE 'CREATE TEMPORARY TABLE} @> intermDst1 AS SELECT DISTINCT(dst) FROM ' || sTable || '
    WHERE src <@\textcolor{red} {IN (SELECT * FROM intermDst)}'@>;
\end{lstlisting}
Die \textcolor{blue}{zweite temporäre Tabelle} beinhaltet die Spalte $dst$ der aktuellen Rekursionsstufe.
Diese wird anschließend in ein \textcolor{red}{Array} umgewandelt.
\begin{lstlisting}[language=SQL,caption = Erstellen des Aufrufarray,frame=single, label={2.AbbruchbedingungRecursiveSearch.listing} ]
    intermDst_ := <@\textcolor{red}{ARRAY}@>(SELECT * FROM <@\textcolor{blue}{intermDst1}@>);
\end{lstlisting}
$intermDst$\_ wird als \textcolor{red}{Aufrufparameter} für die nächste Rekursionsstufe mitgegeben. Darüber hinaus wird die nächst \textcolor{purple}{niedrigere Rekursionsstufe} und die
\textcolor{blue}{Tabelle} als Aufrufparameter an die \textcolor{orange}{Funktion} übergeben:
\begin{lstlisting}[language=SQL,caption = Aufrufen der nächst tieferen Rekursionsstufe,frame=single, label={2.AbbruchbedingungRecursiveSearch.listing} ]
    return query SELECT * FROM <@\textcolor{orange}{recursivesearch}@>(<@\textcolor{red}{intermDst\_}@>, <@\textcolor{purple}{iRecursionDepth - 1}@>, <@\textcolor{blue}{sTable}@>);
\end{lstlisting}
In der nächst tieferen Rekursionsstufe dient die zweite temporäre Tabelle als die Tabelle, die alle $src$ Spalten der aktuellen Rekursionsstufe beinhaltet.
Die zweite temporäre Tabelle wird mit Hilfe von dynamischen \ac{SQL} erstellt.
Es wird Dynamisches SQL verwendet, da das \ac{SQL} auf jeder Rekursionsstufe mit anderen Parametern neu generiert wird.
Dynamisches SQL ermöglicht die Erstellung von Programmen, die in gewisser Weise generisch oder allgemeingültig sind und somit die wechselnden Parameter der verschiedenen Rekursionsstufen verarbeiten können.\cite[S.316 - 317]{froehlich01}

Zuerst wurde zur Erstellung des Skriptes anstelle einer temporären Tabelle eine standard Tabelle verwendet.
Dadurch war das Stored Procedure jedoch um den Faktor sieben langsamer.
Es ist die Vermutung, dass durch die Anlage als temporäre Tabelle, die Tabelle im Shared Memory angelegt wird und so ein Performancegewinn erzielt wird \cite[S.26]{froehlich01}
%\lstsetsql
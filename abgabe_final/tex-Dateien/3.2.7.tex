\subsection{Graphtraversierung mit Hilfe von Standard \ac{SQL}}
\label{2.postgresStandardSQL.subsection}
Bei der Graphtraversierung mit Hilfe von Standard \ac{SQL} wird der Befehl WITH $RECURSIVE$ und $UNION$ verwendet.
Im Anhang befindet sich ein SQL-Skript, welches die Tabelle relation$\_$facebook mit Hilfe des $WITH RECURSIVE$ Befehls bis zur Rekursionsstufe fünf traversiert (siehe \ref{2.StandardSQL.listing}),
sowie ein Skript, welches den Standard SQL Source Code dynamisch generiert (siehe \ref{2.StandardSQLGenerisch.listing}).
Der $WITH$ Befehl erstellt eine temporäre Tabelle, die nur für die angegebene Query existiert.
Diese werden oft auch als \ac{CTE} bezeichnet.
Der SQL Befehl $RECURSIVE$ sorgt dafür, dass die Abfrage sich mit der Ergebnismenge wieder selber aufruft.
Die Struktur einer $RECURSIVE$ Query sieht folgedermaßen aus:
Zuerst wird der \textcolor{blue}{nicht rekursive Teil} der Query angegeben, anschließend wird der \textcolor{orange}{rekursive Teil} beschrieben.
\lstset{escapeinside={<@}{@>}}
\lstsetsql
\begin{lstlisting}[language=SQL,caption = Rekursiver und nicht rekursiver Teil,frame=single, label={2.StrukturderQuery.listing} ]
    WITH RECURSIVE graphtraverse(src, dst, lvl) AS(
    <@\textcolor{blue}{SELECT src ,dst, 1 as lvl FROM public.relation$\_$facebook WHERE src =765} @>
    UNION
    <@\textcolor{orange}{SELECT p1.src,p1.dst,p.lvl+1 as lvl FROM graphtraverse p, relation$\_$facebook p1 WHERE p1.src IN ( p.dst ) and lvl<5} @>
    ) SELECT DISTINCT(dst) FROM graphtraverse order by dst;
\end{lstlisting}
Die Rekursion operiert hierzu auf zwei temporären Tabellen.
Der $Working$ Tabelle und der $Intermediate$ Tabelle.
Ein Rekursionsschritt sieht folgendermaßen aus:
Die Ergebnisse innerhalb einer Rekursionsstufe werden in die $Working$ Tabelle geschrieben, es wird mit Hilfe des $UNION$ Operators überprüft ob Duplikate vorhanden sind.
Die Überprüfung erfolgt bezogen auf eine Rekursionsstufe.
Duplikate werden gelöscht, die Ergebissmenge wird in die Intermediate Tabelle geschrieben, die Ergebnisse aus der Intermediate Tabelle werden in die $Working$ Tabelle kopiert, die $Intermediate$ Tabelle wird gelöscht \cite{postgreswithrecursive}.
Die Abbruchbedingung für die Rekursion wird innerhalb der Funktion definiert.
Ob die Abbruchbedingung erreicht ist, wird im rekursiven Teil der Funktion \textcolor{red}{überprüft}:
\begin{lstlisting}[language=SQL,caption = Überprüfen der Abbruchbedingung,frame=single, label={2.Abbruchbedingung.listing} ]
    SELECT p1.src,p1.dst,p.lvl+1 as lvl FROM graphtraverse p, relation$\_$facebook p1 WHERE p1.src IN ( p.dst ) and <@\textcolor{red}{lvl<5} @>
\end{lstlisting}
Der Ausführungsplan für die Tabelle relation$\_$facebook für den $WITH RECURSIVE$ Operator befindet sich im Anhang (siehe \ref{2.AusführungsplanCTEFacebook.listing}).
Das Verknüpfen der zurückgegebenen Ergebnismenge von der graphtraverse Funktion mit der relation$\_$facebook Tabelle erfolgt in einem \textcolor{blue}{Nested Loop Join}.
Ein Nested Loop Join besteht aus zwei Tabellen über die iteriert wird.
Die erste Tabelle (Outer Table) wird einmal durchlaufen.
Die zweite Tabelle (Inner Table) wird mehrfach durchlaufen.
Der Vergleich wird über eine Indexsuche gebildet \cite[Seite 213]{froehlich01}.
Die \textcolor{red}{Gleichheitsüberprüfung} (src = p.dst) wird mit Hilfe eines Index Scans gemacht.
%\footnote{Bei einem Index-Scan ist die Index-Zugriffsmethode dafür verantwortlich, die \ac{TID}s aller Tupel, von denen ihr mitgeteilt wurde, dass sie mit den Scan-Schlüsseln übereinstimmen, wiederherzustellen.}
Die Überprüfung ob das Rekursionslevel erreicht worden ist, erfolgt mit Hilfe eines \textcolor{green}{Scans} über die $Working$ Table:
%\newpage
\begin{lstlisting}[language=SQL,caption = Überprüfung der WHERE Bedingung,frame=single, label={2.WhereConditionCTE.listing} ]
    ->  <@\textcolor{blue}{Nested Loop}@>  (cost=0.29..1434.82 rows=2111 width=12) (actual time=0.018..2.174 rows=8173 loops=5)
    ->  WorkTable <@\textcolor{green}{Scan}@> on graphtraverse p  (cost=0.00..5.17 rows=77 width=8) (actual time=0.016..0.065 rows=729 loops=5)
    Filter: (lvl < 5)
    Rows Removed by Filter: 482
    ->  Index Scan using indexsrc on relation_facebook p1  (cost=0.29..18.23 rows=27 width=8) (actual time=0.001..0.002 rows=11 loops=3645)
    <@\textcolor{red}{Index Cond: (src = p.dst)}@>
\end{lstlisting}
Der $RECURSIVE$ Operator und der $UNION$ Operator finden sich in der Zeile Recursive Union wieder.
Dieser Teil der Ausführung dauert am längsten (16,071 ms).
\begin{lstlisting}[language=SQL,caption = Aufruf RECURSIVE und UNION Operator,frame=single, label={2.WhereConditionCTE.listing} ]
    ->  Recursive Union  (cost=0.29..14814.87 rows=21133 width=12) (actual time=0.014..16.085 rows=6056 loops=1)
\end{lstlisting}
\newpage
Der Aufruf der graphtraverse Funktion findet sich in der letzten Zeile des Ausführungsplan:
\begin{lstlisting}[language=SQL,caption = Aufruf der graphtraverse Funktion,frame=single, label={2.WhereConditionCTE.listing} ]
    SELECT DISTINCT(dst) FROM <@\textcolor{blue}{graphtraverse}@> order by dst
\end{lstlisting}

\begin{lstlisting}[language=SQL,caption = Aufruf der graphtraverse Funktion im Ausführungsplan,frame=single, label={2.functionCallGraphtraverse.listing} ]
    ->  CTE Scan on <@\textcolor{blue}{graphtraverse}@>  (cost=0.00..422.66 rows=21133 width=4) (actual time=0.015..16.921 rows=6056 loops=1)
\end{lstlisting}
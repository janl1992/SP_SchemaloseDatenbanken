\section{Modell}
Allgemein betrachtet ist ein Modell eine vereinfachte bzw. abstrahierte Darstellung von realen Gegenständen, Sachverhalten oder Problemen.
Durch die Modellierung soll die Realität auf die wichtigsten Einflussfaktoren reduziert werden \cite{datamodels}.
In der Datenbankwelt beschreibt das Modell die Struktur der Daten, Operationen zum manipulieren der Daten und Integritätsbedingungen \cite{efcodd}.
Das derzeit am häufigsten verwendete Modell ist das relationale Datenbankmodell, bei dem die Daten in Tabellen gespeichert werden und in der Regel die standardisierte Abfragesprache \ac{SQL} eingesetzt wird.
Der Schwachpunkt des relationalen Datenbankmodells liegt bei der Verarbeitung von Daten mit einer hohen Anzahl an Beziehungen \cite{vicknair2010comparison}.

%Die Graphentheorie spielt eine zentrale Rolle bei der Modellierung, da sich Graphen sehr gut zur Darstellung vernetzter Daten eignen.
Da eine effiziente Verarbeitung von großen vernetzten Datenmengen immer wichtiger wird, haben Graphenmodelle in den letzten Jahren im Datenbankbereich stark an Bedeutung gewonnen.
Graph-Datenbanken nutzen Graphen als Datenbankmodell und greifen auf graphenspezifische Operationen zur effizienten Verarbeitung vernetzter Daten zurück \cite{angles2008survey}.
%Obwohl sich die verschiedenen Graphdatenbanken im allgemeinen nur minimal bei der Strukturierung der Daten unterscheiden, gibt es sehr verschiedene Ansätze bei den Abfragesprachen \cite{anglesintro}.
%Dennoch sind die kleinen Unterschiede in der Datenmodellierung oft entscheidend und es können je nach Anwendungsfall verschiedene Modelle sinnvoll sein \cite{angles2012comparison}.
Trotz der oft kleinen Unterschiede in der Datenmodellierung können je nach Anwendungsfall verschiedene Modelle sinnvoll sein \cite{angles2012comparison}.
In der Praxis werden oft verschiedene Modelle zu Multi-Model Datenbanken kombiniert, um die Schwachpunkte der einzelnen Modelle auszugleichen.
Ein Beispiel für eine solche Multi-Model Datenbank ist OrientDB, welche unteranderem das Graphen-Modell mit Key-Value Stores und dem Objektorientierten-Modell verbindet \cite{orient}.
Im Folgenden sollen verschiedene Graphdatenbankmodelle, sowie Aspekte der Graphentheorie, die zur Modellierung von Graphdatenbanken von Bedeutung sind, kurz vorgestellt werden.
\newpage
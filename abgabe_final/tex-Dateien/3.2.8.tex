\subsection{Graphtraversierung mit Hilfe von verschachteltem SELECT Statement}
Bei der Graphtraversierung mit Hilfe von verschachteltem \ac{SQL} wird ein selbsterstelltes verschachteltes $SELECT$ Statement verwendet.
Ein Beispielstatement befindet sich im Anhang (siehe \ref{2.SELECT.listing}).
Auf der obersten Rekursionsstufe wird der Startknoten des Graphen mitgegeben (in diesem Beispiel ist der Startknoten = 1).
Das Ergebnis dieser Abfrage wird als Eingabe für die nächst tiefere Rekursionsstufe verwendet.
In der $WHERE$ Klausel wird für die Spalte $src$ der $IN$ Operator verwendet.
Der $IN$ Operator erlaubt es, mehrere Werte innerhalb der $WHERE$ Klausel anzugeben.
Das $DISTINCT$ in der $SELECT$ Klausel sorgt dafür, dass Duplikate in der Ergebnismenge der momentanen Rekursionsstufe entfernt werden.
Die Funktionsweise von $DISTINCT$ ist in der folgenden Grafik nochmal dargestellt:
\begin{figure}[H]
    \includegraphics[width = \linewidth]{images/{2.Distinct}.jpg};
    %    \label{2.duplikate.image}
    \caption{Löschen von Duplikaten in einer Rekursionsstufe}
\end{figure}
Hierbei liegt der Knoten fünf so, dass er in der zweiten Rekursionsstufe zwei Mal in der Auswahl auftaucht.
$DISTINCT$ entfernt das Duplikat.
Die Ergebnismenge, entfernt um die Duplikate, wird als Input für die nächste Rekursionsstufe verwendet.
Die Ausgabe des verschachtelten $SELECT$ Statement sind die Nachbarn der Knoten, der angegebenen Rekursionstiefe.
Wird zum Beispiel ein verschachteltes $SELECT$ Statement der Tiefe drei erstellt, so gibt dieses Statement alle Nachbarn dritten Grades ausgehend vom Startknoten an.
Der Nachteil bei dieser Methode ist, dass Kreise in einem Graph nicht erkannt werden.
Die Duplikatüberprüfung erfolgt nicht über mehrere Rekursionsstufen hinweg, sondern immer nur zwischen zwei Rekursionsstufen.
Der Ausführungsplan für ein verschachteltes $SELECT$ der Tiefe fünf befindet sich im Anhang (siehe \ref{2.AusführungsplanCascadeSELECT.listing}).
Für die $DISTINCT$ Überprüfung wird auf jeder Rekursionsebene ein HashAggregate herangezogen.
Die \textcolor{red}{Zugehörigkeitsüberprüfung} zur $IN$ Klausel in der $WHERE$ Bedingung ist in folgendem Listing gegeben:
\begin{lstlisting}[language=SQL,caption = IN Klausel,frame=single, label={2.INKlauselFacebook.listing} ]
    SELECT DISTINCT(dst) FROM relation_facebook WHERE src <@\textcolor{red}{IN}@> ()
\end{lstlisting}
Die Zugehörigkeitsüberprüfung erfolgt auf allen Rekursionsstufen, ausgenommen der ersten Rekursionsstufe, mit Hilfe eines \textcolor{red}{Hash Join}.
Bevor der Hash Join durchgeführt wird, wird zuerst der \textcolor{blue}{Hash} gebildet und anschließend ein \textcolor{orange}{Sequential Scan} durchgeführt:
\begin{lstlisting}[language=SQL,caption = Aufruf der DISTINCT Funktion,frame=single, label={2.WhereConditionCTE.listing} ]
    ->  <@\textcolor{red}{Hash Join}@>  (cost=2713.40..4389.64 rows=88234 width=4) (actual time=11.821..17.797 rows=1709 loops=1)
    ->  <@\textcolor{orange}{Seq Scan}@> on relation_facebook relation_facebook_1  (cost=10000000000.00..10000001649.62 rows=100762 width=8) (actual time=0.004..4.822 rows=100762 loops=1)
    ->  <@\textcolor{blue}{Hash}@>  (cost=60.87..60.87 rows=19 width=4) (actual time=0.022..0.022 rows=4 loops=1)
\end{lstlisting}
Auf der ersten Rekursionsstufe erfolgt die Überprüfung der $IN$ Klausel mit Hilfe eines \textcolor{red}{Index Scans}:
\begin{lstlisting}[language=SQL,caption = IndexScanFacebookRelation,frame=single, label={2.indexScanFacebookRelation.listing} ]
    <@\textcolor{red}{Index Scan}@> using indexsrc on
    relation_facebook relation_facebook_4 (cost =0.29..44.00 rows =23 width =4) ( actual time =0.009..0.012 rows =27 loops =1)
    Index Cond: (src = 765)
\end{lstlisting}
Dieser wird nur auf der ersten Rekursionsstufe verwendet, weil nur hier auf einer physischen Tabelle operiert wird.
Auf den tieferen Rekursionsstufen wird immer ein Sequential Scan vollzogen (vergleiche \ref{2.AusführungsplanCascadeSELECT.listing}).
Hier wird immer auf einer Zwischentabelle operiert, die mit Hilfe des $IN$ Operators erstellet wird.
Auf diesen Tabellen operiert das verschachtelte $SELECT$ Statement jedoch die meiste Zeit, weshalb die meiste Zeit ein sequential Scan verwendet wird.


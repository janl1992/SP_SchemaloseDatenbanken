\subsection{Visitenkarte des Systems}
\begin{itemize}
    \item Allgemein
    \begin{itemize}
        \item Name: PostgreSQL, umgangssprachlich Postgres
        \item Kategorie / Modell: PostgreSQL ist ein Relationales Datenbank System
        \item Version: 11.1
        \item Historie: PostgreSQL ist aus dem POSTGRES Projekt der University of California at Berkeley entstanden, welches unter der Leitung von  Professor Michael Stonebraker im Jahre 1986 began.
        \item Hersteller: PostgreSQL Global Development Group
        \item Lizenz: Open-Source
        \item Referenzen / Quellenangaben: \cite{froehlich01}, \cite{postgres2018}, \cite{postgresqldoc}, \cite{eisentraut01}
    \end{itemize}
    \item Architektur
    \begin{itemize}
        \item Programmiersprache: C
        \item Systemarchitektur: Objektrelationales Datenbankmanagementsystem
        \item Betriebsart: Stand Alone, Cluster Betrieb für Replikation der Datenbank
        %            \item Protokoll der Schnittstelle: TCP/IP % \cite[Seite 28]{froehlich01}
        \item \ac{API}: u.A. libpq, psycopg, psqlODBC, pq, pgtclng, Npgsql, node-postgres
    \end{itemize}
    \newpage
    \item Datenmodell
    \begin{itemize}
        \item Standardsprache: PL/pgSQL % \cite[Seite 270]{froehlich01}
        %            \item Objektbegriffe, Konzepte:
        \item Sichten (Views): Ja
        \item Externe Dateien (BLOBs): Ja
        \item Schlüssel: Ja
        \item Semantische unterschiedliche Beziehungen: Ja
        \item Constraints: Ja
    \end{itemize}
    \item Indexe
    \begin{itemize}
        \item Sekundärindexe: Ja
        \item Gespeicherte Prozeduren: Ja
        \item Triggermechanismen: Ja, Prozeduren , die als Trigger aufgerufen werden % \cite[Seite 610]{eisentraut01}
        \item Versionierung: Ja, Versionierung mit Hilfe von Transaktions-ID(XID) % \cite[Seite 38]{froehlich01}
    \end{itemize}
    \item Anfragemethoden
    \begin{itemize}
        \item Kommunikation: \ac{SQL} über \ac{TCP}/\ac{IP}
        \item \ac{CRUD}-Operationen: Ja
        \item Ad-hoc-Anfragen: Ja
        %            \item Kopplungstechniken Programmiersprache:
        %            \item Map/Reduce:
    \end{itemize}
    %        Horizontale Skalierbarkeit fällt weg
    \item Konsistenz
    \begin{itemize}
        \item \ac{ACID}, besoners \ac{MVCC}
        \item Transaktionen: Ja
        \item Nebenläufigkeit (Synchronisation): Ja
        %            \item Dauerhaftigkeit: Ja, weil ACID konform
        %            \item Konfliktbehandlung Replikation:
    \end{itemize}
    \item Administration
    \begin{itemize}
        \item Werkzeuge: pgAdmin, dataGrip, diverse Erweiterungen
        \item Massendatenimport: Ja
        \item Datensicherung: Ja
        %            \item Recovery: Ja
    \end{itemize}
\end{itemize}
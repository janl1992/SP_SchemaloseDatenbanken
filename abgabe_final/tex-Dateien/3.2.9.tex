\subsection{Graphtraversierung mit Hilfe von rekursiven INNER JOIN}
\label{2.postgresInnerJoin.subsection}
Bei der Graphtraversierung mit Hilfe von rekursiven $INNER JOIN$ soll der Graph traversiert werden, indem die Relationentabelle immer wieder mit sich selber verknüpft wird.
Ausgegebnen werden, ähnlich wie bei der Graphtraversierung mit Hilfe von verschachteltem SELECT Statement, die Nachbarn der Knoten, die sich auf der mitgegebenen Rekursionsstiefe befinden.
Ein Beispielstatement für den rekursiven $INNER JOIN$ ist im Anhang gegeben (siehe \ref{2.JOIN.listing}).
Der Ausführungsplan für die relation$\_$facebook Tabelle befindet sich ebenfalls im Anhang (\ref{2.AusführungsplanINNERJOIN.listing}).
Fast alle Verknüpfungen werden mit Hilfe eines Hash Join vollzogen.
Bei dem Ausführungsplan fällt der Merge Join auf, der sehr viel Zeit in Ansprich nimmt:
\begin{lstlisting}[language=SQL,caption = Merge JOIN,frame=single, label={2.mergeJoin.listing} ]
    ->  Merge Join  (cost=89342.23..356052.39 rows=17747686 width=4) (actual time=112.509..1178.121 rows=8863706 loops=1)
\end{lstlisting}
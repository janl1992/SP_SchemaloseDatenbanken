\section{PostgreSQL}
Die Implementierung von OLTP Anwendungsfällen ist eine klassische Aufgabe für relationale Datenbanken.
Da statt einer Graphdatenbank eine relationale Datenbank verwendet wurde, ist die Implementierung des OLTP Anwendungsfalls (Gästebuch) uninteressant.
Interessant ist jedoch die Umsetzung der Traversierung von Graphen in relationalen Datenbanken.
Für die Graphtraversierung wurden eigene Scripte geschrieben, die in diesem Kapitel vorgestellt werden.
%\subsection{Traversierung in PostgresSQL}
%Das Ziel ist es 5 Graphen mit Hilfe einer objektrelationalen Datenbank zu traversieren.
Da Graphdatenbanken für das Traversieren von Graphen entwickelt worden sind, sollten diese bei der Traversierung einen Performancegewinn gegenüber objektrelationalen Datenbanken haben.
%Die Vermutung ist, dass das Traversieren eines Graphen mit Hilfe einer objektrelationalen Datenbank ähnlich performant ist, wie das Traversieren mit Hilfe einer Graphdatenbank.
Ziel ist es, mit Hilfe einer objektrelationalen Datenbank eine mit den Graphdatenbanken vergleichbar performante Abfrage eines Graphen zu implementieren.
Für die Graphtraversierung sind die folgenden 5 Methoden vorgesehen.

Graphtraversierung mit Hilfe von:
\begin{itemize}
    \item Rekursiven \ac{CTE}
    \item Verschachteltem SELECT Statement
    \item Rekursiven INNER JOIN
    \item Selbstgeschriebenen Stored Procedure
    \item Dynamisch generiertem \ac{SQL}
\end{itemize}
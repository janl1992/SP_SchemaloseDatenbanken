\subsection{Indexe und Partionierte Tabellen}
Um eine bessere Aussage über die Performance von PostgreSQL zu bekommen wurden zusätzlich zu den ''Default'' --Relationstabellen für jeden Datensatz zwei weitere Tabellen angelegt.

Die erste Tabelle hat das Suffix ''\_with\_index''. Im Listing ist Beispielhaft das Create-Skript für die Relationen aus dem Datensatz ''youtube'' dargestellt:

\lstsetsql
\begin{lstlisting}[language=SQL,caption = Tabelle mit Index anlegen,frame=single, label={2.tabelleIndex.listing} ]
CREATE TABLE IF NOT EXISTS relation_youtube_with_index(
src INTEGER REFERENCES profiles_youtube(ID),
dst INTEGER REFERENCES profiles_youtube(ID),
type VARCHAR(50),
date DATE
);
CREATE INDEX yt_dst ON relation_youtube_with_index (dst);
CREATE INDEX yt_src ON relation_youtube_with_index (src);
\end{lstlisting}

Im Unterschied zur ''public\_youtube''--Tabelle wurde diese Tabelle noch um Zwei Indices erweitert. Dabei wurde ein index auf die src-Spalte angelegt, der andere auf die dst-Spalte.


Neben der den ''\_with\_index''--Tabellen wurden noch die \_partitioned--Tabellen angelegt. Diese verwenden die gleichen Indices wie die ''\_with\_index''-Tabellen, jedoch sind sie zusätzlich in 4 Partitonen aufgteilt:

\begin{lstlisting}[language=SQL,caption = Partitonierte Tabelle mit Indices anlegen,frame=single, label={2.tabelleIndex.listing} ]

CREATE TABLE IF NOT EXISTS relation_youtube_partitioned(
src INTEGER REFERENCES profiles_youtube(ID),
dst INTEGER REFERENCES profiles_youtube(ID),
type VARCHAR(50),
date DATE
)PARTITION BY RANGE(src);
CREATE INDEX yt_part_src ON relation_youtube_partitioned (src);
CREATE INDEX yt_part_dst ON relation_youtube_partitioned (dst);
CREATE TABLE relation_youtube_partitioned_0 PARTITION OF relation_youtube_partitioned
FOR VALUES FROM (0) TO (800000);
CREATE TABLE relation_youtube_partitioned_1 PARTITION OF relation_youtube_partitioned
FOR VALUES FROM (800001) TO (1600000);
CREATE TABLE relation_youtube_partitioned_2 PARTITION OF relation_youtube_partitioned
FOR VALUES FROM (1600001) TO (2400000);
CREATE TABLE relation_youtube_partitioned_3 PARTITION OF relation_youtube_partitioned
FOR VALUES FROM (2400001) TO (3200000);
\end{lstlisting}

Da die Datensätze unterschiedlich groß sind wurden die Partionsgrößen entsprechend angepasst. Jeder Datensatz wurde auf 4 Partitonen verteilt. Damit ergibt sich für den Datensatz youtube eine Partitonsgröße von 800.000.

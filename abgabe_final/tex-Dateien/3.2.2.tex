\subsection{Installation von Postgres}
PostgreSQL kann unter Ubuntu über die Paketverwaltung \ac{APT} installiert werden.
Weiterhin wird eine Installation über die \ac{RPM}-Paketverwaltung angeboten.
Im Rahmen dieser Arbeit wird PostgreSQL Version 11 verwendet.
Ein Parallelbetrieb verschiedener PostgreSQL Versionen ist möglich.
Nach der Installation von PostgreSQL muss zunächst der Befehl $initdb$ ausgeführt werden.
Über $initdb$ wird ein PostgreSQL-Cluster angelegt.
Als Parameter kann ein Directory-Pfad angegeben werden.
In diesem Pfad wird der PostgreSQL-Cluster von $initdb$ angelegt.
Gemäß der Vorgaben dieser Arbeit wurde das PostgreSQL-Cluster unter $/data/team22/postgresql/11/main$ installiert.
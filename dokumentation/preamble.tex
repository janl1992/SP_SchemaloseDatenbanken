% \renewcommand\abstractname{Danksagung}
\begin{abstract}
\section*{Vorwort}\markboth{Vorwort}{}
  \addcontentsline{toc}{chapter}{Vorwort}
\enlargethispage*{\baselineskip}
%\begin{center}
\begin{figure}[H]
  \begin{table}[H]
  \centering
    \begin{tabular}{|c|p{2.5cm}|l|l|l|}
      \hline
      \textbf{Kapitel} & \textbf{schriftlich} & \textbf{Umsetzung} & \textbf{Vortrag erstellt} & \textbf{Vortrag gehalten}\\
      \hline
      \hline
      1 & Wittling & - & Wittling & Wittling \\
      \hline
      %2 & \multicolumn{2}{|l}{Löffelsender, Wittling} \vline & - & - \\
      2 & Löffelsender, \newline Wittling & - & - & - \\
      \hline
      3 & Löffelsender, \newline Wittling & Löffelsender & Löffelsender & Löffelsender\\
      \hline
      4 & Kimmelmann & Kimmelmann & Kimmelmann & Kimmelmann \\
      \hline
      4 & Kimmelmann & - & - & - \\
      \hline
      \multicolumn{2}{|c}{Systemadministration} \vline & Kimmelmann & - & -\\
      \hline
      %5 & sed diam voluptua & 1005 \\
      %\hline
      %6 & clita kasd gubergren & 1006 \\
      %\hline
    \end{tabular}
    \caption{Aufgabenverteilung}
%    \label{table1}
  \end{table}
\end{figure}
%\end{center}
  Die packages
  \begin{itemize}
    \item amsthm,
    \item lstautogobble,
    \item multirow
  \end{itemize}
  wurden zusaätzlich zur Standardvorlage verwendet.

\end{abstract}

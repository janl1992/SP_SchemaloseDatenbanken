\chapter{Graph-Datenbanken im praktischen Einsatz: OLAP}
\section{PostgreSQL: OLAP}
\subsection{Benchmark}
Mit der Standardinstallation von PostgreSQL wird auch pgbench mitinstalliert. Bei pgbench handelt es sich um ein einfaches Tool zur Durchführung von Benchmark-Tests. Bei einem Benchmark-Test wird eine Menge von SQL-Statements beliebig oft wiederholt, dabei können auch mehrere parallele Sessions geöffnet werden. Beim durchführen des Tests berechnet pgbench die Anzahl der Transaktionen pro Sekunde.
\subsubsection{Verwendung von pgbench}
pgbench wird über die Kommandozeile gestartet. Dabei können eine Reihe von Parametern übergeben werden, mit denen das Verhalten von pgbench gesteuert werden kann.
\begin{itemize}
	\item -c clients  \\
	Über das Flag -c wird die Anzahl der Clients bzw. die Anzahl der gleichzeitigen Datenbankverbindungen festgelegt. Wenn hier nichts angegeben ist wird nur ein Client verewendet.
	\item -t transactions \\
	Über das Flag -t wird festgelegt wieviele Transaktionen jeder Client durchführt. Die Anzahl aller Transaktionen ergibt sich durch das Produkt von Clients und Transactions.
\end{itemize}
\subsection{Standard SQL}
\subsection{Stored Procedures}
\subsection{PL/SQL-Recursion}
\subsection{Datenbankzugriffe}
\subsection{Zugriffsart Aggregation}
\subsection{Zugriffsart Traversierung}
\subsection{Interpretation der Ergebnisse}
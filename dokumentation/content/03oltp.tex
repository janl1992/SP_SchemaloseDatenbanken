\chapter{Graph-Datenbanken im praktischen Einsatz: OLTP}
\section{PostgresSQL: OLTP}
\subsection{Installation}
PostgreSQL kann unter Ubuntu über die Paketverwaltung APT installiert werden. Weiterhin wird eine Installation über die RPM-Paketverwaltung angeboten. Im Rahmen dieser Arbeit wird PostgreSQL Version 11 verwendet. Ein Parallelbetrieb verschiedener PostgreSQL Versionen ist möglich. 
Nach der Installation von PostgreSQL muss zunächst $initdb$ ausgeführt werden. Über $initdb$ wird ein PostgreSQL-Cluster angelegt. Als Parameter kann ein Directory-Pfad angegeben werden. In diesem Pfad wird der PostgreSQL-Cluster von $initdb$ angelegt.
Gemäß der Vorgaben dieser Arbeit wurde das PostgreSQL-Cluster unter $/data/team22/postgresql/11/main$ installiert.
\subsection{Ausgewählte Use Cases}
\subsection{CSV-Import}
Beim Import von (CSV)-Dateien wird zwischen import vom Clientsystem und  Import vom Serversystem unterschieden. 
Für den Import vom Client wird das psql-Statement \textbackslash copy verwendet. \textbackslash copy liest Informationen aus einer Datei die vom psql-Client aus erreichbar sein muss. \cite{postgres2018}

\begin{lstlisting}[language=SQL,caption=CSV Input,frame=single]
\copy Beitraege 
FROM './data/Beitraege.csv' DELIMITER ',' CSV HEADER;
\end{lstlisting}


\begin{lstlisting}[language=SQL,caption=Anlegen der Tabelle facebook-profiles,frame=single]
    CREATE TABLE public."facebook-profiles"
    (
        id serial PRIMARY KEY,
        first TEXT,
        last TEXT,
        gender TEXT,
        country TEXT,
        birth TEXT
    );
\end{lstlisting}

\begin{lstlisting}[language=SQL,caption=Anlegen der Tabelle facebook,frame=single]
    CREATE TABLE public.facebook
    (
        src INT,
        dst INT,
        type TEXT,
        date TEXT
    );
\end{lstlisting}

\begin{lstlisting}[language=SQL,caption=Hinzufügen von Fremdschlüsseln,frame=single]
    ALTER TABLE public.facebook
    ADD CONSTRAINT "facebook_facebook-profiles_id_fk"
    FOREIGN KEY (src) REFERENCES public."facebook-profiles" (id)
\end{lstlisting}

\subsection{Beurteilung}
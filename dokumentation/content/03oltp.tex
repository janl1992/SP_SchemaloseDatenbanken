\chapter{Graph-Datenbanken im praktischen Einsatz: OLTP}
\section{PostgresSQL: OLTP}
\subsection{Ausgewählte Use Cases}
\subsection{CSV-Import}
Beim Import von (CSV)-Dateien wird zwischen import vom Clientsystem und  Import vom Serversystem unterschieden. 
Für den Import vom Client wird das psql-Statement \textbackslash copy verwendet. \textbackslash copy liest Informationen aus einer Datei die vom psql-Client aus erreichbar sein muss. \cite{postgres2018}

\begin{lstlisting}[language=SQL,caption=CSV Input,frame=single]
\copy Beitraege 
FROM './data/Beitraege.csv' DELIMITER ',' CSV HEADER;
\end{lstlisting}


\begin{lstlisting}[language=SQL,caption=Anlegen der Tabelle facebook-profiles,frame=single]
    CREATE TABLE public."facebook-profiles"
    (
        id serial PRIMARY KEY,
        first TEXT,
        last TEXT,
        gender TEXT,
        country TEXT,
        birth TEXT
    );
\end{lstlisting}
\newpage
\begin{lstlisting}[language=SQL,caption=Anlegen der Tabelle facebook,frame=single]
    CREATE TABLE public.facebook
    (
        src INT,
        dst INT,
        type TEXT,
        date TEXT
    );
\end{lstlisting}

\begin{lstlisting}[language=SQL,caption=Hinzufügen von Fremdschlüsseln,frame=single]
    ALTER TABLE public.facebook
    ADD CONSTRAINT "facebook_facebook-profiles_id_fk"
    FOREIGN KEY (src) REFERENCES public."facebook-profiles" (id)
\end{lstlisting}

\subsection{Beurteilung}
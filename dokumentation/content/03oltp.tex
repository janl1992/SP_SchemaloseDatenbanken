\chapter{Graph-Datenbanken im praktischen Einsatz: OLTP}
\section{PostgresSQL: OLTP}
\subsection{Installation}
PostgreSQL kann unter Ubuntu über die Paketverwaltung APT installiert werden. Weiterhin wird eine Installation über die RPM-Paketverwaltung angeboten. Im Rahmen dieser Arbeit wird PostgreSQL Version 11 verwendet. Ein Parallelbetrieb verschiedener PostgreSQL Versionen ist möglich. 
Nach der Installation von PostgreSQL muss zunächst $initdb$ ausgeführt werden. Über $initdb$ wird ein PostgreSQL-Cluster angelegt. Als Parameter kann ein Directory-Pfad angegeben werden. In diesem Pfad wird der PostgreSQL-Cluster von $initdb$ angelegt.
Gemäß der Vorgaben dieser Arbeit wurde das PostgreSQL-Cluster unter $/data/team22/postgresql/11/main$ installiert.
\subsection{Ausgewählte Use Cases}
\subsection{CSV-Import}
Beim Import von (CSV)-Dateien wird zwischen import vom Clientsystem und  Import vom Serversystem unterschieden. 
Für den Import vom Client wird das psql-Statement \textbackslash copy verwendet. \textbackslash copy liest Informationen aus einer Datei die vom psql-Client aus erreichbar sein muss. \cite{postgres2018}


\subsection{Beurteilung}
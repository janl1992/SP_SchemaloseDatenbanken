\chapter{Graph-Datenbanken und -Frameworks - Ausgewählte Systeme }
\section{PostgreSQL}
\subsection{Visitenkarte des Systems}
    \begin{itemize}
        \item Allgemein
        \begin{itemize}
            \item Name: PostgreSQL, umgangssprachlich Postgres
            \item Kategorie / Modell: PostgreSQL ist ein Relationales Datenbank System
            \item Version: 11.1
            \item Historie: PostgreSQL ist aus dem POSTGRES Projekt der University of California at Berkeley entstanden, welches unter der Leitung von  Professor Michael Stonebraker im Jahre 1986 began.
            \item Hersteller: PostgreSQL Global Development Group
            \item Lizenz: Open-Source
            \item Referenzen / Quellenangaben: \cite{froehlich01}, \cite{postgres2018}, \cite{postgresqldoc}, \cite{eisentraut01}
        \end{itemize}
        \item Architektur
        \begin{itemize}
            \item Programmiersprache: C
            \item Systemarchitektur: Objektrelationales Datenbankmanagementsystem
            \item Betriebsart: Stand Alone, Cluster Betrieb für Replikation der Datenbank
%            \item Protokoll der Schnittstelle: TCP/IP % \cite[Seite 28]{froehlich01}
            \item \ac{API}: u.A. libpq, psycopg, psqlODBC, pq, pgtclng, Npgsql, node-postgres
        \end{itemize}
        \newpage
        \item Datenmodell
        \begin{itemize}
            \item Standardsprache: PL/pgSQL % \cite[Seite 270]{froehlich01}
%            \item Objektbegriffe, Konzepte:
            \item Sichten (Views): Ja
            \item Externe Dateien (BLOBs): Ja
            \item Schlüssel: Ja
            \item Semantische unterschiedliche Beziehungen: Ja
            \item Constraints: Ja
        \end{itemize}
        \item Indexe
        \begin{itemize}
            \item Sekundärindexe: Ja
            \item Gespeicherte Prozeduren: Ja
            \item Triggermechanismen: Ja, Prozeduren , die als Trigger aufgerufen werden % \cite[Seite 610]{eisentraut01}
            \item Versionierung: Ja, Versionierung mit Hilfe von Transaktions-ID(XID) % \cite[Seite 38]{froehlich01}
        \end{itemize}
        \item Anfragemethoden
        \begin{itemize}
            \item Kommunikation: \ac{SQL} über \ac{TCP}/\ac{IP}
            \item \ac{CRUD}-Operationen: Ja
            \item Ad-hoc-Anfragen: Ja
%            \item Kopplungstechniken Programmiersprache:
%            \item Map/Reduce:
        \end{itemize}
%        Horizontale Skalierbarkeit fällt weg
        \item Konsistenz
        \begin{itemize}
            \item \ac{ACID}, besoners \ac{MVCC}
            \item Transaktionen: Ja
            \item Nebenläufigkeit (Synchronisation): Ja
%            \item Dauerhaftigkeit: Ja, weil ACID konform
%            \item Konfliktbehandlung Replikation:
        \end{itemize}
        \item Administration
        \begin{itemize}
            \item Werkzeuge: pgAdmin, dataGrip, diverse Erweiterungen
            \item Massendatenimport: Ja
            \item Datensicherung: Ja
%            \item Recovery: Ja
        \end{itemize}
    \end{itemize}
\newpage
%PostgreSQL ist eine Objektrelationale Datenbank. Weiterentwicklungen wird von der PostgreSQL Global Development Group durchgeführt. PostgreSQL steht unter der PostgreSQL-Lizenz, welche sehr stark der GNU-Lizenz ähnelt. Auf Basis von PostgreSQL gibt es mehrere, zum Teil auch kommerzielle, Forks wie zum Beispiel Amazon Redshift oder EnterpriseDB.
%\subsection{Architektur}
%    \begin{itemize}
%        \item Programmiersprache des Systems
%        \item Systemkomponenten, Systemarchitektur
%        \subitem client / server Architektur
%        \item Betriebsart
%            \subitem Cluster - Menge an Datenbanken, die von PostSQL-Server verwaltet werden \cite{froehlich01}
%        \item Protokoll der Schnittstelle
%        \subitem TCP/IP
%        \item API
%    \end{itemize}
%    Die Architektur von PostgresSQL ist in folgendem Bild gegeben:
%    \begin{center}
%        \includegraphics[width = \linewidth]{./images/PostgresSQLArchitektur.jpg}
%    \end{center}
%    Diese Architektur ergibt sich aus der Gegebenheit, dass alle Daten normalerweise nicht in den Shared Buffer passen. Ein Teil der Datenbank befindet sich im Datafile, aus dem bei Bedarf gelesen werden kann, der andere Teil im Shared Buffer.
%    Die Hauptaufgabe des Shared Buffer ist es, Input Output Operationen (I/O Operationen) auf das Datafile zu minimieren und möglichst viele Operationen im Speicher durchzuführen. Die Motivation möglichst viele Operation im Speicher durchzuführen
%    besteht darin, dass Operationen im Speicher schneller ausgeführt werden. Werden Operationen im Speicher ausgeführt, so werden I/O Operationen auf das Datafile reduziert. Der Writer Prozess ist für die Synchronisation
%    des Zustands des Shared Buffers und der Tablespaces verantwortlich. Dieser Prozess schreibt Datenblöcke aus dem Shared Buffer auf die Tablespaces innerhalb des Datafile. Neben diesem writer Prozess gibt es noch andere BackenProzesse, die in dem Bild unter BackendProzess zusammengefasst sind
%    Der Writer Checkpoint ist dafür verantwortlich, dass alle geänderten Datenblöcke innerhalb des Shared Buffer in das Datafile geschrieben werden. Der WAL (Write-Ahead Log) Buffer, der WAL Writer sowie die WAL-Datei bilden das Transaktionslog des Clusters. Sie werden genauer im Kapitel Konsistenz beschrieben.
%    \footnote{Vgl. \cite[Seite 26]{froehlich01}} VACUUM beschreibt den Prozess, der Teil der Verwaltung mehrere Versionen
%    von Datenbankblöcken ist. In Postgres werden verschiedene Versionen von Datensätzen in der Tabelle gespeichert. Ältere Versionen von den Datensätzen werden gelöscht, wenn sie nicht mehr benötigt werden. Dadurch entstehen
%    Lücken in der Tabelle, die der Vacum Prozess als nicht mehr benötigt kennzeichnet. \footnote{Vgl. \cite[Seite 37]{froehlich01}}
%
%    Eine PostgreSQL-Instanz wird als Server Prozess mit eigenem Datenverzeichnis und einer eigenen Konfigurationsdatei sowie einem eigenen Transaktionslog.
%\subsection{Datenmodell}
%    \begin{itemize}
%        \item Standardsprache: SQL
%%        \item Objektbegriff, Konzepte: Relationale Datenbank - Abbildung in Tabellen
%        \item Datentypen:
%        \subitem Sehr viele unterstützte Datentypen, es lassen sich aber auch eigene Datentypen mittesl des CREATE TYPE befehls erstellen.\cite{postgres8}
%    \end{itemize}
%    Daten werden in Form von Tabellen abgelegt. Im Filesystem legt PostgreSQL Dateien im sogenannten \$PGDATA-Verzeichnis ab, dieses wird beim Start von PostgreSQL festgelegt. Alle Date werden im Verzeichnis Base unterhalb von \$PGDATA abgelegt. Einzelne Tabellen oder Indices können in Tablespaces ausgelagert werden. Für einen Tablespace wird ein neuer Unterordner im \$PGDATA-Verzeichnis angelegt. Es ist auch möglich einzelne Spalten einer Tabelle in einen anderen Tablespace zu verschieben.
%\subsection{Konsistenz}
%PostgreSQL schreibt einen Transaktionslog (Write-Ahead Log, WAL).
%Dieser wird als WAL-Buffer im Arbeitsspeicher und als WAL-Dateien auf der Festplatte geführt.
%Bei jedem Commit einer Transaktion wird zunächst das WAL aktualisiert bevor die Bestätigung an den Client gesendet wird.
%Der walwriter-Prozess schreibt periodisch die WAL-Buffer auf die Festplatte.
%\subsection{Indexe}
%\subsection{Anfragemethoden}
%\subsection{Konsistenz}